\documentclass[a4paper]{jarticle}
\usepackage{sf}


\title{センシングフォーラム予稿テンプレート}
\author{○金谷 孝一郎$^1$, 山川 雄司$^2$}
\affiliation{
$^1$ 東京大学,
$^2$ 東京大学 }
\Etitle{Template for proceedings of Sensing Forum}
\Eauthor{○Koichiro Kanaya and Yuji Yamakawa}
\Eaffiliation{
$^1$ University of Tokyo,
$^2$ University of Tokyo }
\Abstract{日本語アブストラクト(本文が英語の場合は英語でも可).
200字程度で記載.
〇〇〇〇〇〇〇〇〇〇〇〇〇〇〇〇〇〇〇〇〇〇〇〇〇〇〇〇〇〇〇〇〇〇
〇〇〇〇〇〇〇〇〇〇〇〇〇〇〇〇〇〇〇〇〇〇〇〇〇〇〇〇〇〇〇〇〇〇
〇〇〇〇〇〇〇〇〇〇〇〇〇〇〇〇〇〇〇〇〇〇〇〇〇〇〇〇〇〇〇〇〇〇
〇〇〇〇〇〇〇〇〇〇〇〇〇〇〇〇〇〇〇〇〇〇〇〇〇〇〇〇〇〇〇〇〇〇
〇〇〇〇〇〇〇〇〇〇〇〇〇〇〇〇〇〇〇〇〇〇〇〇〇〇〇〇〇〇〇〇〇〇
}
\begin{document}
\maketitle
\section{原稿の書き方}
\begin{itemize}
\item 原稿枚数はA4版で4~6ページです.超過しないようご注意下さい.
\item 用紙余白は上下24mm,左右15mmとし,本文を縦250mm×横180mmの枠内に収めて下さい.
\item 冒頭に以下の項目を書いてください.
\begin{itemize}
\item 一行目:和文題目.
\item 二行目:和文著者名.登壇者の前に必ず○をつけてください.
\item 三行目:和文所属名.
\item 四行目:英文題目.
\item 五行目:英文著者名.登壇者の前に必ず○をつけてください.
\item 六行目:英文所属名.
\item 七行目以降:要旨(日本語.論文の本文が英文の場合,英語でも結構です)
\end{itemize}
\item 原稿はPDF形式で作成してください.印字の正確性を期すため,PDFファイル作成時にフォントの埋め込みをお願い致します.
\item 引用は文献\cite{ref1}のように記載してください.
\end{itemize}
\begin{thebibliography}{9}
\bibitem{ref1}
計測 太郎:センシングフォーラム予稿の書き方,
第39回センシングフォーラム論文集,pp. 1-5, 2022.
\end{thebibliography}
\end{document}
% end of sftemplate.tex